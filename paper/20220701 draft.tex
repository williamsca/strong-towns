\documentclass[12pt]{article}
\usepackage[right=1.25in,left=1.25in,top=1.1in,bottom=1.1in]{geometry}
\usepackage{hyperref}
\hypersetup{colorlinks, citecolor=blue, filecolor=blue, linkcolor=blue, urlcolor=blue}
\usepackage{graphicx}
\usepackage{url}
\usepackage[round]{natbib}
\usepackage{amsmath,amsthm} 
\usepackage{engord}
\usepackage{float}
\usepackage{subfig}
\usepackage{pdflscape}
\usepackage{natbib}
\usepackage{booktabs}
\usepackage{pgfplots}

\usepackage{epigraph} % used for opening quotation
\setlength{\epigraphwidth}{\textwidth}

\pgfplotsset{compat=1.14}
\pgfplotsset{every axis label/.append style={font=\tiny}}
\usepackage[labelsep=period]{caption} %% This switches "Table 1: Title" to "Table 1. Title"

\usepackage{amssymb} %% Necessary, just for the \checkmark command  in tables.
% \usepackage{multirow} %% Necessary if we are doing tables in LaTeX

\usepackage{xr}

\usepackage{setspace}
\onehalfspacing

\usepackage{sectsty}
\sectionfont{\large}
\subsectionfont{\normalsize}
\subsubsectionfont{\normalsize}

\newcommand{\specialcell}[2][c]{\begin{tabular}[#1]{@{}l@{}}#2\end{tabular}}

%%%%%%%%%%%%%%%%%%%%%%%%%%%%%%%%%%%%%%%%%%%%%%%%%%%%%%%%%%%%%

\title{ \vspace*{-2.5cm} \hspace*{-0.5cm}Sprawling for Reelection: \\ The Public Choice of Low-Density Development \footnote{
...
}}

\author{Colin Williams\thanks{University of Virginia.
\href{mailto:chv7bg@virginia.edu}{chv7bg@virginia.edu}} }%\and Author Two\thanks{TK University and
%NBER.  \href{mailto:TK@TK.edu}{TK@TK.edu}} \and Author Three\thanks{TK
% University. \href{mailto:TK@TK.edu}{TK@TK.edu}}}

\date{ \vspace*{0.5cm} July 2022 \\
\textbf{Preliminary and Incomplete. \\ Please do not cite or circulate.}
} 

%%%%%%%%%%%%%%%%%%%%%%%%%%%%%%%%%%%%%%%%%%%%%%%%%%%%%%%%%%%%%

\begin{document}

\bgroup
\let\footnoterule\relax

\begin{singlespace}
\maketitle


\begin{abstract}
    \noindent Local officials choose if and how to permit new development. I argue that the nature of infrastructure costs, which require an initial outlay and only limited maintenance expenses for several decades, allows myopic local officials to raise tax revenues in the present while deferring replacement costs to their successors by privileging new construction. Using plausibly exogenous variation in local government finances induced by a federal revenue sharing program running from 1972 through 1986,  
\end{abstract}
\end{singlespace}
\thispagestyle{empty}

\clearpage
\egroup
\setcounter{page}{1}

%% Temporary tool to track how this paper is structured. Feel free to comment in or out. 
% \tableofcontents
% \bigskip

%%%%%%%%%%%%%%%%%%%%%%%%%%%%%%%%%%%%%%%%%%%%%%%%%%%%%%%%%%%%%
%%%%\section{Introduction\label{sec:introduction}}

\noindent % We also cite papers in this document \citep{Chetty2013}. For instance: \citet{Hansen1992}. So on and so forth\ldots

% The remainder of the paper proceeds as follows. Section \ref{sec:background} provides background. We then present our
% empirical results in Section \ref{sec:results}. Finally, Section \ref{sec:conclusion} concludes. 

\epigraph{The proud minister of an ostentatious court may frequently take pleasure in executing a work of splendor and magnificence, such as a great high-way which is frequently seen by the principal nobility... But to execute a great number of little works, in which nothing that can be done can make any great appearance, or excite the smallest degree of admiration in any traveller... is a business which appears in every respect too mean and paltry to merit the attention of so great a magistrate. Upon such an administration, therefore, such works are almost always neglected.}{\textit{Adam Smith \\ The Wealth of Nations (V.i.d.16)}}

\section{Background \label{sec:background}}

The costs of new infrastructure rarely fall directly on users. As noted by \citet{brueckner2000urban}, new roads and sewers are typically financed by taxes on all city residents rather than being paid by those living in the development which occasioned the expense. In effect, taxpayers pay the average cost of infrastructure across the city rather than the marginal cost of their use. This market failure tends to result in overly large cities.

Our argument is slightly different: rather than passing the cost of new infrastructure onto current residents, officials exploit non-linear maintenance schedules alongside federal infrastructure grants to ``pull forward'' revenues while passing on the costs to their successors.

\section{A Model of Local }

A local official maximizes her probability of re-election by choice of a per-capita tax rate $\tau$ and the number of new construction permits to approve $d$. The official assesses taxes on $N$ current residents and $d$ new residents, yielding a total revenue of $(1 + \tau)(N+d)$. For now, $N$ is exogenous. \citep{knight_endogenous_2002}

Tax revenues are used to maintain the infrastructure -- e.g., local roads and sewerage -- that services current residents at cost $k$ per resident. New construction is funded by impact fees assessed on the developer and is costless from the perspective of the official.\footnote{Impact fees, or exactions, proliferated rapidly during the 1970s as local governments sought to ensure that new developments paid their own way \citet{altshuler_regulation_2000}.}


\section{Results \label{sec:results}}

% \input{tab_tex/summary_stats.tex}


% Table \ref{tab:summary_statistics}

% \input{fig_tex/fig_mainsummary.tex}

% Figure \ref{fig:mainsummary} 

% \input{tab_tex/regressions.tex}

% The top panel of Table \ref{tab:regressions} 

\section{Conclusion\label{sec:conclusion}}


%%%%%%%%%%%%%%%%%%%%%%%%%%%%%%%%%%%%%%%%%%%%%%%%%
\clearpage
\begin{singlespace}
% \bibliographystyle{plainnat}
\bibliographystyle{chicago}
%\bibliographystyle{aer}
\bibliography{citations.bib}
\end{singlespace}
%%%%%%%%%%%%%%%%%%%%%%%%%%%%%%%%%%%%%%%%%%%%%%%%%




\end{document}