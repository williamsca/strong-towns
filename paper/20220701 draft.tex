\documentclass[12pt]{article}
\usepackage[right=1.25in,left=1.25in,top=1.1in,bottom=1.1in]{geometry}
\usepackage{hyperref}
\hypersetup{colorlinks, citecolor=blue, filecolor=blue, linkcolor=blue, urlcolor=blue}
\usepackage{graphicx}
\usepackage{url}
\usepackage[round]{natbib}
\usepackage{amsmath,amsthm} 
\usepackage{engord}
\usepackage{float}
\usepackage{subfig}
\usepackage{pdflscape}
\usepackage{natbib}
\usepackage{booktabs}
\usepackage{pgfplots}
\pgfplotsset{compat=1.14}
\pgfplotsset{every axis label/.append style={font=\tiny}}
\usepackage[labelsep=period]{caption} %% This switches "Table 1: Title" to "Table 1. Title"

\usepackage{amssymb} %% Necessary, just for the \checkmark command  in tables.
% \usepackage{multirow} %% Necessary if we are doing tables in LaTeX

\usepackage{xr}

\usepackage{setspace}
\onehalfspacing

\usepackage{sectsty}
\sectionfont{\large}
\subsectionfont{\normalsize}
\subsubsectionfont{\normalsize}

\newcommand{\specialcell}[2][c]{\begin{tabular}[#1]{@{}l@{}}#2\end{tabular}}

%%%%%%%%%%%%%%%%%%%%%%%%%%%%%%%%%%%%%%%%%%%%%%%%%%%%%%%%%%%%%

\title{ \vspace*{-2.5cm} \hspace*{-0.5cm}Sprawling for Reelection: \\ The Public Choice of Low-Density Development \footnote{
...
}}

\author{Colin Williams\thanks{University of Virginia.
\href{mailto:chv7bg@virginia.edu}{chv7bg@virginia.edu}} \and Author Two\thanks{TK University and
NBER.  \href{mailto:TK@TK.edu}{TK@TK.edu}} \and Author Three\thanks{TK
University. \href{mailto:TK@TK.edu}{TK@TK.edu}}}

\date{ \vspace*{0.5cm} July 2022 \\
\textbf{Preliminary and Incomplete. \\ Please do not cite or circulate.}
} 

%%%%%%%%%%%%%%%%%%%%%%%%%%%%%%%%%%%%%%%%%%%%%%%%%%%%%%%%%%%%%

\begin{document}

\bgroup
\let\footnoterule\relax

\begin{singlespace}
\maketitle


\begin{abstract}
    \noindent Local officials choose if and how to permit new development. We argue that the nature of infrastructure costs, which require an initial outlay and only limited maintenance expenses for several decades, allows myopic local officials to raise tax revenues in the present while deferring replacement costs to their successors. Officials can maximize immediate revenues by approving low-density expansion. Consistent with this mechanism, we find evidence that cities constrained by political or topographical boundaries have higher tax rates and higher municipal debt.
\end{abstract}
\end{singlespace}
\thispagestyle{empty}

\clearpage
\egroup
\setcounter{page}{1}

%% Temporary tool to track how this paper is structured. Feel free to comment in or out. 
% \tableofcontents
% \bigskip

%%%%%%%%%%%%%%%%%%%%%%%%%%%%%%%%%%%%%%%%%%%%%%%%%%%%%%%%%%%%%
%%%%\section{Introduction\label{sec:introduction}}

\noindent % We also cite papers in this document \citep{Chetty2013}. For instance: \citet{Hansen1992}. So on and so forth\ldots

% The remainder of the paper proceeds as follows. Section \ref{sec:background} provides background. We then present our
% empirical results in Section \ref{sec:results}. Finally, Section \ref{sec:conclusion} concludes. 

\section{Background \label{sec:background}}

The costs of new infrastructure rarely fall directly on users. As noted by \citet{brueckner2000urban}, new roads and sewers are typically financed by taxes on all city residents rather than being paid by those living in the development which occasioned the expense. In effect, taxpayers pay the average cost of infrastructure across the city rather than the marginal cost of their use. This market failure tends to  in overly large cities.

Our argument is slightly different: rather than passing the cost of new infrastructure onto current residents, officials exploit non-linear maintenance schedules alongside federal infrastructure grants to ``pull forward'' revenues while passing on the costs to their successors.



\section{Results \label{sec:results}}

% \input{tab_tex/summary_stats.tex}


% Table \ref{tab:summary_statistics}

% \input{fig_tex/fig_mainsummary.tex}

% Figure \ref{fig:mainsummary} 

% \input{tab_tex/regressions.tex}

% The top panel of Table \ref{tab:regressions} 

\section{Conclusion\label{sec:conclusion}}


%%%%%%%%%%%%%%%%%%%%%%%%%%%%%%%%%%%%%%%%%%%%%%%%%
\clearpage
\begin{singlespace}
\bibliographystyle{plainnat}
%\bibliographystyle{chicago}
%\bibliographystyle{aer}
\bibliography{citations.bib}
\end{singlespace}
%%%%%%%%%%%%%%%%%%%%%%%%%%%%%%%%%%%%%%%%%%%%%%%%%


%%%%%%%%%%%%%%%%%%%%%%%%%%%%%%%%%%%%%%%%%%%%%%%%%
%%%%% These commands start the appendix and change the Table & Figure numbering
\newpage
\appendix
\setcounter{table}{0}
\renewcommand{\tablename}{Appendix Table}
\renewcommand{\figurename}{Appendix Figure}
\renewcommand{\thetable}{A\arabic{table}}
\setcounter{figure}{0}
\renewcommand{\thefigure}{A\arabic{figure}}
%%%%%%%%%%%%%%%%%%%%%%%%%%%%%%%%%%%%%%%%%%%%%%%%%

\section{Appendix Tables and Figures}
% \input{tab_tex/other-regressions.tex}

\newpage 
\section{Appendix One \label{sec:appendix:first}}
\renewcommand{\thetable}{B\arabic{table}}
\setcounter{table}{0}
\renewcommand{\thefigure}{B\arabic{figure}}
\setcounter{figure}{0}

% \input{fig_tex/fig_another_figure.tex}

\newpage
\section{Appendix Two
\label{sec:appendix:two}}
\renewcommand{\thetable}{C\arabic{table}}
\setcounter{table}{0}
\renewcommand{\thefigure}{C\arabic{figure}}
\setcounter{figure}{0}


\end{document}